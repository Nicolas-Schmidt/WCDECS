% Options for packages loaded elsewhere
\PassOptionsToPackage{unicode}{hyperref}
\PassOptionsToPackage{hyphens}{url}
\PassOptionsToPackage{dvipsnames,svgnames*,x11names*}{xcolor}
%
\documentclass[
  ignorenonframetext,
]{beamer}
\usepackage{pgfpages}
\setbeamertemplate{caption}[numbered]
\setbeamertemplate{caption label separator}{: }
\setbeamercolor{caption name}{fg=normal text.fg}
\beamertemplatenavigationsymbolsempty
% Prevent slide breaks in the middle of a paragraph
\widowpenalties 1 10000
\raggedbottom
\setbeamertemplate{part page}{
  \centering
  \begin{beamercolorbox}[sep=16pt,center]{part title}
    \usebeamerfont{part title}\insertpart\par
  \end{beamercolorbox}
}
\setbeamertemplate{section page}{
  \centering
  \begin{beamercolorbox}[sep=12pt,center]{part title}
    \usebeamerfont{section title}\insertsection\par
  \end{beamercolorbox}
}
\setbeamertemplate{subsection page}{
  \centering
  \begin{beamercolorbox}[sep=8pt,center]{part title}
    \usebeamerfont{subsection title}\insertsubsection\par
  \end{beamercolorbox}
}
\AtBeginPart{
  \frame{\partpage}
}
\AtBeginSection{
  \ifbibliography
  \else
    \frame{\sectionpage}
  \fi
}
\AtBeginSubsection{
  \frame{\subsectionpage}
}
\usepackage{lmodern}
\usepackage{amssymb,amsmath}
\usepackage{ifxetex,ifluatex}
\ifnum 0\ifxetex 1\fi\ifluatex 1\fi=0 % if pdftex
  \usepackage[T1]{fontenc}
  \usepackage[utf8]{inputenc}
  \usepackage{textcomp} % provide euro and other symbols
\else % if luatex or xetex
  \usepackage{unicode-math}
  \defaultfontfeatures{Scale=MatchLowercase}
  \defaultfontfeatures[\rmfamily]{Ligatures=TeX,Scale=1}
\fi
% Use upquote if available, for straight quotes in verbatim environments
\IfFileExists{upquote.sty}{\usepackage{upquote}}{}
\IfFileExists{microtype.sty}{% use microtype if available
  \usepackage[]{microtype}
  \UseMicrotypeSet[protrusion]{basicmath} % disable protrusion for tt fonts
}{}
\makeatletter
\@ifundefined{KOMAClassName}{% if non-KOMA class
  \IfFileExists{parskip.sty}{%
    \usepackage{parskip}
  }{% else
    \setlength{\parindent}{0pt}
    \setlength{\parskip}{6pt plus 2pt minus 1pt}}
}{% if KOMA class
  \KOMAoptions{parskip=half}}
\makeatother
\usepackage{xcolor}
\IfFileExists{xurl.sty}{\usepackage{xurl}}{} % add URL line breaks if available
\IfFileExists{bookmark.sty}{\usepackage{bookmark}}{\usepackage{hyperref}}
\hypersetup{
  pdftitle={Proceso y etapas del diseño de investigación: Preguntas y credibilidad},
  pdfauthor={Diseño e implementación de experimentos en ciencias sociales},
  colorlinks=true,
  linkcolor=Maroon,
  filecolor=Maroon,
  citecolor=Blue,
  urlcolor=Blue,
  pdfcreator={LaTeX via pandoc}}
\urlstyle{same} % disable monospaced font for URLs
\newif\ifbibliography
\setlength{\emergencystretch}{3em} % prevent overfull lines
\providecommand{\tightlist}{%
  \setlength{\itemsep}{0pt}\setlength{\parskip}{0pt}}
\setcounter{secnumdepth}{-\maxdimen} % remove section numbering
\setbeamertemplate{footline}{\begin{beamercolorbox}{section in head/foot}
\insertframenumber/\inserttotalframenumber \end{beamercolorbox}}


\title{Proceso y etapas del diseño de investigación: Preguntas y
credibilidad}
\author{Diseño e implementación de experimentos en ciencias sociales}
\date{}

\begin{document}
\frame{\titlepage}

\begin{frame}[allowframebreaks]
  \tableofcontents[hideallsubsections]
\end{frame}
\hypertarget{el-proceso-y-las-etapas-del-diseuxf1o-de-investigaciuxf3n}{%
\section{El Proceso y las etapas del diseño de
investigación}\label{el-proceso-y-las-etapas-del-diseuxf1o-de-investigaciuxf3n}}

\begin{frame}{¿Qué hace que una pregunta de investigación sea buena?}
\protect\hypertarget{quuxe9-hace-que-una-pregunta-de-investigaciuxf3n-sea-buena}{}
\begin{itemize}
\item
  La respuesta a una buena pregunta de investigación debe generar
  conocimiento que le interese a la gente.
\item
  Abordar la pregunta debería (ayudar a) resolver un problema, tomar una
  decisión o aclarar/desafiar nuestro entendimiento del mundo.
\item
  Pero una pregunta interesante no es suficiente
\end{itemize}
\end{frame}

\begin{frame}{También necesitamos un buen diseño de investigación}
\protect\hypertarget{tambiuxe9n-necesitamos-un-buen-diseuxf1o-de-investigaciuxf3n}{}
\begin{itemize}
\item
  Un buen diseño de investigación es un plan práctico de investigación
  que aprovecha los recursos disponibles y produce una respuesta
  creíble.
\item
  La calidad de un diseño de investigación se puede evaluar por los
  resultados que produce, los cuales pueden utilizarse para orientar las
  políticas y contribuir a la ciencia:

  \begin{itemize}
  \item
    Un excelente diseño de investigación produce resultados que nos
    guían claramente en ciertas direcciones que son relevantes.
  \item
    Un diseño de investigación deficiente produce resultados que nos
    dejan en la oscuridad, o sea, resultados cuya interpretación es
    confusa.
  \end{itemize}
\end{itemize}
\end{frame}

\begin{frame}[allowframebreaks]{La importancia de la teoría}
\protect\hypertarget{la-importancia-de-la-teoruxeda}{}
Todo diseño de investigación involucra una teoría, ya sea de manera
implícita o explícita.

\begin{itemize}
\item
  ¿Por qué investigar? Todos tenemos teorías y valores implícitos que
  guían las preguntas que hacemos. Nuestras preguntas reflejan nuestros
  valores. Por ejemplo, en la década de los 1950 científicos sociales
  estudiaron el consumo de marihuana como una forma de ``desviación'',
  las preguntas se centraron en ``¿por qué la gente está tomando
  decisiones tan malas?'' o ``¿cómo pueden los legisladores prevenir el
  consumo de marihuana?'' (ver Howards y Becker 1998).
\item
  ¿Por qué investigar? Puede que queramos cambiar la forma en la que la
  ciencia explica el mundo y/o cambiar las decisiones de políticas para
  (a) un lugar y un momento en el tiempo y/o (b) para otros lugares y
  momentos en el tiempo.
\item
  La investigación que está centrada en conocer el efecto causal de
  \(X\) en \(Y\) requiere pensar en un modelo del mundo: \emph{cómo} la
  intervención \(X\) podría afectar a una variable de resultado \(Y\), y
  \emph{por qué} y \emph{qué tan grande} podría ser el efecto. Esto nos
  ayuda a pensar en cómo una intervención diferente o que esté dirigida
  a diferentes destinatarios podría producir resultados diferentes.
\item
  Nuestras teorías y modelos no son solo importantes para generar
  hipótesis, sino también para informar \emph{el diseño} y \emph{las
  estrategias de inferencia}.
\item
  El diseño de investigación a menudo mostrará dónde tenemos menos
  certeza acerca de nuestras teorías. Nuestras teorías nos indicarán
  dónde hay problemas con nuestro diseño. Y las preguntas que surjan del
  proceso de diseño pueden dejarnos ver la necesidad de trabajar más en
  la explicación y el mecanismo.
\end{itemize}
\end{frame}

\begin{frame}{Diseñando o escogiendo el tratamiento}
\protect\hypertarget{diseuxf1ando-o-escogiendo-el-tratamiento}{}
\begin{itemize}
\item
  Desde este punto en adelante, usaremos \(T\) para el tratamiento o el
  efecto en el que estamos interesados. Usaremos \(X\) para referirnos a
  las variables de contexto.
\item
  El tratamiento (\(T\)) y control (la ausencia de \(T\)) deben estar
  claramente relacionados a la pregunta de investigación. (Ver el módulo
  de Medición).
\item
  Un tratamiento de interés puede ser un conjunto con varios
  componentes. Si su pregunta de investigación es sobre un componente
  específico, entonces el control debe ser diferente del tratamiento
  solo en ese componente. Todo lo demás debe ser igual.
\end{itemize}
\end{frame}

\begin{frame}{Un ejemplo}
\protect\hypertarget{un-ejemplo}{}
Una campaña de salubridad en la que un voluntario visita un hogar para
hablar con una familia durante 15 minutos para dar información sobre
temas de salud.

\begin{itemize}
\item
  Si se tiene interés en el efecto de la información en específico,
  entonces el control debe incluir todos los demás componentes (visita
  domiciliaria con una duración de 15 minutos, voluntario con
  características similares, etc.) pero variar la información que se
  comparte. Este diseño no nos enseñará el efecto de las visitas, sino
  solo el efecto de la información en sí.
\item
  Si su pregunta se centra en el efecto de las visitas, entonces se
  necesita un grupo de control que no reciba visitas. Sin embargo este
  diseño no servirá para responder preguntas específicas sobre el
  contenido de la información (las visitas y la información están
  combinadas en un sólo tratamiento).
\end{itemize}
\end{frame}

\begin{frame}{Interpretación}
\protect\hypertarget{interpretaciuxf3n}{}
\begin{itemize}
\item
  A veces no es posible aislar un componente específico del tratamiento.
\item
  Por ejemplo, puede ser que la organización de salud comunitaria con la
  que están colaborando para visitar hogares no esté interesada en
  visitar hogares y compartir otra información. En ese caso el control
  podría ser no visitar.
\item
  Se debe tener cuidado en interpretar los resultados como el efecto de
  la información comunicada de esta manera en particular.
\item
  No se puede concluir que se ha estimado solamente el efecto de la
  información comunicada.

  \begin{itemize}
  \item
    Puede que esto sea adecuado para ciertas políticas: quizá la
    pregunta de la política pública sea del efecto de las visitas como
    parte de una combinación implícita de tratamientos.
  \item
    Pero es difícil interpretar los resultados de este diseño si
    queremos saber algo específico sobre el efecto de la información
    únicamente.
  \end{itemize}
\end{itemize}
\end{frame}

\hypertarget{el-proceso-de-investigaciuxf3n}{%
\section{El Proceso de
investigación}\label{el-proceso-de-investigaciuxf3n}}

\begin{frame}{Una descripción general del proceso de investigación}
\protect\hypertarget{una-descripciuxf3n-general-del-proceso-de-investigaciuxf3n}{}
\begin{itemize}
\item
  Articular y afinar la pregunta (preguntándonos la razón de hacer esta
  pregunta y qué sucedería si obtenemos diferentes tipos de respuestas).
\item
  Desarrollar el diseño de investigación.
\item
  Planificar el análisis, presentar y justificar hipótesis específicas,
  así como registrar el plan con un sello impersonal creíble señalando
  la fecha del registro.
\item
  Implementar la intervención y recopilar los datos.
\item
  Analizar los datos y escribir los resultados
\end{itemize}
\end{frame}

\hypertarget{el-formulario-para-el-diseuxf1o-de-investigaciuxf3n-de-egap}{%
\section{El Formulario para el diseño de investigación de
EGAP}\label{el-formulario-para-el-diseuxf1o-de-investigaciuxf3n-de-egap}}

\begin{frame}{El Formulario para el diseño de investigación de EGAP}
\protect\hypertarget{el-formulario-para-el-diseuxf1o-de-investigaciuxf3n-de-egap-1}{}
\begin{itemize}
\item
  Considerar el
  \href{https://egap.github.io/learningdays-resources/Exercises/design-form.Rmd}{Formulario
  para el diseño de investigación} que provee de una estructura para
  desarrollar un buen diseño de investigación.

  \begin{itemize}
  \tightlist
  \item
    \url{https://egap.github.io/learningdays-resources/Exercises/design-form.Rmd}
  \end{itemize}
\item
  Les puede ayudar a:

  \begin{itemize}
  \item
    escribir una propuesta de investigación para solicitar financiación
    para un proyecto, y/o
  \item
    desarrollar un plan de pre-análisis.
  \end{itemize}
\end{itemize}
\end{frame}

\begin{frame}{Secciones del formulario para el diseño de investigación
EGAP}
\protect\hypertarget{secciones-del-formulario-para-el-diseuxf1o-de-investigaciuxf3n-egap}{}
\begin{itemize}
\tightlist
\item
  Pregunta de investigación
\item
  Muestra
\item
  Tratamiento
\item
  variable de resultado
\item
  Estrategia de asignación aleatoria
\item
  Implementación
\item
  Poder estadístico
\item
  Análisis e interpretación
\end{itemize}
\end{frame}

\begin{frame}{Pregunta de investigación y motivación}
\protect\hypertarget{pregunta-de-investigaciuxf3n-y-motivaciuxf3n}{}
\begin{itemize}
\item
  ¿Cuál es la motivación sustantiva de esta investigación? ¿Qué problema
  se está tratando de abordar? ¿Qué decisión se está tratando de tomar?
\item
  ¿A quiénes están tratando de cambiar de parecer? y ¿qué creen esas
  personas actualmente?
\item
  ¿Qué preguntas teóricas se pueden abordar con esta investigación?
\item
  Concretar la pregunta de investigación en una sola oración.
\item
  ¿Cuál es la hipótesis principal?
\end{itemize}
\end{frame}

\begin{frame}{Muestra}
\protect\hypertarget{muestra}{}
\begin{itemize}
\item
  ¿Dónde y cuándo tendrá lugar el estudio?
\item
  ¿Quiénes o cuáles son las unidades de investigación en el estudio?
\item
  ¿Cómo se selecciona la muestra?
\item
  ¿Es necesario dejar fuera del estudio a algunas unidades porque deben
  recibir el tratamiento? O por razones logísticas o de otro tipo
  ¿algunas unidades deben quedar por fuera?
\item
  ¿Se espera que el tratamiento funcione de manera diferente para
  ciertos subgrupos?
\end{itemize}
\end{frame}

\begin{frame}{Tratamiento}
\protect\hypertarget{tratamiento}{}
\begin{itemize}
\item
  ¿Cuál es el tratamiento? ¿Habrá múltiples tratamientos?
\item
  ¿Cuál es el control? ¿Control puro o placebo?
\item
  ¿Existe alguna preocupación ética relacionada al tratamiento?
\item
  ¿A qué nivel se realizará la asignación aleatoria del tratamiento?
\end{itemize}
\end{frame}

\begin{frame}{Variable de resultado}
\protect\hypertarget{variable-de-resultado}{}
\begin{itemize}
\item
  ¿Cuál es la variable de resultado principal?
\item
  ¿Cómo se medirá?
\item
  ¿Qué datos se necesitan? ¿A qué nivel se pueden medir los datos?
\item
  ¿Qué ideas tienen sobre la variable de resultado previo a la
  recolección de datos (priors)? Éstas pueden basarse en estudios
  previos o suposiciones bien fundamentadas.
\item
  ¿Cuántas rondas de recolección de datos habrá?
\item
  ¿Cómo se minimizará la deserción?
\item
  ¿Cómo se minimizará la mala medición y las respuestas deshonestas?
\end{itemize}
\end{frame}

\begin{frame}{Estrategia de aleatorización}
\protect\hypertarget{estrategia-de-aleatorizaciuxf3n}{}
\begin{itemize}
\item
  ¿Qué tipo de estrategia de asignación aleatoria utilizarán? Ejemplos:
  simple, completo, bloques, conglomerados, factorial, de dos niveles,
  por etapas, etc.
\item
  Asegúrense de que esta estrategia sea coherente con el nivel al que se
  está aleatorizando (posiblemente conglomerados) y la heterogeneidad
  que se espera en los efectos del tratamiento (posiblemente bloques).
\item
  Definan sus bloques y conglomerados (si los hay). ¿Cuántos van a
  tener? ¿Qué tan grandes serán?
\item
  ¿Es posible que haya interferencia? Si es así, ¿cómo les ayudará su
  muestra y la estrategia de aleatorización a minimizar la
  interferencia?
\end{itemize}
\end{frame}

\begin{frame}{Implementación I}
\protect\hypertarget{implementaciuxf3n-i}{}
\begin{itemize}
\item
  ¿Cómo llevará a cabo la asignación aleatoria en la práctica? ¿En
  público, sacando una ficha de una urna? ¿En una computadora?
\item
  ¿Quién estará a cargo de implementar el tratamiento?
\item
  Si es un socio el que implementará el tratamiento, ¿qué acuerdos
  tienen?
\item
  ¿Cuáles son los desafíos logísticos? ¿Algún desafío especial para las
  unidades de control?
\end{itemize}
\end{frame}

\begin{frame}{Implementación II}
\protect\hypertarget{implementaciuxf3n-ii}{}
\begin{itemize}
\item
  ¿Cómo dará seguimiento a la calidad de la implementación?
\item
  ¿Cómo dará seguimiento al cumplimiento del tratamiento?
\item
  ¿Cómo minimizará el incumplimiento del tratamiento (si aplica)?
\item
  ¿Cómo comprobará la calidad de sus datos?
\item
  ¿Cómo se anonimizarán y almacenarán los datos de forma segura (si
  aplica)?
\end{itemize}
\end{frame}

\begin{frame}{Poder estadístico}
\protect\hypertarget{poder-estaduxedstico}{}
\begin{itemize}
\item
  ¿Cuál es el tamaño esperado del efecto?

  \begin{itemize}
  \tightlist
  \item
    Esta información puede venir de un estudio anterior o un valor
    objetivo por debajo del cual uno no estaría interesado en futuras
    intervenciones.
  \end{itemize}
\item
  Cálculo del poder.

  \begin{itemize}
  \tightlist
  \item
    Si tiene conglomerados, se debe tener en cuenta la correlación
    dentro de los conglomerados.
  \end{itemize}
\end{itemize}
\end{frame}

\begin{frame}{Análisis e interpretación}
\protect\hypertarget{anuxe1lisis-e-interpretaciuxf3n}{}
\begin{itemize}
\item
  ¿Cuál es el estimado? (p.~ej., efecto promedio del tratamiento, efecto
  causal promedio del cumplidor, etc.)
\item
  ¿Cuál es el estimador? (por ejemplo, diferencia de medias, MCO con
  pesos de bloque, algún conglomerado). Tengan en cuenta que esto
  debería estar estrechamente relacionado con su diseño de
  aleatorización.
\item
  Si encuentra que sus resultados son consistentes con su hipótesis,
  ¿qué explicaciones alternativas podría haber? ¿Qué datos le ayudarían
  a distinguir entre su explicación y otras alternativas?
\item
  Si encuentra que sus resultados no son consistentes con su hipótesis,
  ¿qué datos le ayudarán a averiguar qué pudo haber sucedido?
\end{itemize}
\end{frame}

\hypertarget{declaredesign}{%
\section{DeclareDesign}\label{declaredesign}}

\begin{frame}{Introducción a DeclareDesign}
\protect\hypertarget{introducciuxf3n-a-declaredesign}{}
\begin{itemize}
\item
  Declare Design es un paquete de software en R.
\item
  Nos ayuda a concretar las etapas del diseño de investigación al
  permitirnos representarlas en código, lo que luego nos permitirá
  simular las etapas del diseño de investigación para comprender las
  propiedades de los estimadores estadísticos y las pruebas que
  utilizamos.
\item
  Para obtener más información, consulte
  (\url{https://declaredesign.org/getting-started})
\item
  También pueden consultar el módulo sobre estimaciones y estimadores
  que utiliza DeclareDesign para ayudarles a determinar los estimadores
  apropiados.
\end{itemize}
\end{frame}

\begin{frame}{Introducción a DeclareDesign}
\protect\hypertarget{introducciuxf3n-a-declaredesign-1}{}
\begin{itemize}
\item
  Ver \url{https://declaredesign.org/}
\item
  Independientemente del método, los diseños de investigación tienen
  cuatro componentes
\item
  MIDA:

  \begin{itemize}
  \tightlist
  \item
    M: (Model) Modelo ¿Cómo funciona el mundo?
  \item
    I: (Inquiry) Indagación
  \item
    D: (Data Strategy) Datos ¿Cuál es la estrategia?
  \item
    A: (Answer Strategy) Respuesta ¿Cuál es la estrategia?
  \end{itemize}
\item
  Perspectiva crítica: Simular el diseño de investigación nos puede
  enseñar qué respuestas podemos obtener del diseño de investigación.
\item
  Trabajar con datos simulados \emph{antes de la recopilación de datos}
  nos ayuda a prevenir errores y descuidos.
\end{itemize}
\end{frame}

\begin{frame}{Modelo}
\protect\hypertarget{modelo}{}
\begin{itemize}
\item
  Un modelo de cómo pensamos que el mundo funciona incluye:

  \begin{itemize}
  \item
    \(T\)s y \(X\)s (tratamientos o variables causales centrales como
    intervenciones de políticas y otras variables de contexto)
  \item
    \(Y\)s (variables dependientes)
  \item
    Relaciones entre variables (posibles resultados, relaciones
    funcionales, variables auxiliares y contextos)
  \item
    Distribución de probabilidad sobre \(X\)s y quizá también sobre
    \(Y\)s.
  \end{itemize}
\item
  ¡Esta es la teoría!

  \begin{itemize}
  \tightlist
  \item
    Codificada numéricamente.
  \end{itemize}
\item
  Por definición, los modelos siempre serán incorrectos. Si fueran
  correctos, no necesitaríamos realizar un estudio.
\item
  Pero sin un modelo no tenemos cómo comenzar a evaluar lo que podemos
  aprender.
\end{itemize}
\end{frame}

\begin{frame}{Indagación (Inquiry)}
\protect\hypertarget{indagaciuxf3n-inquiry}{}
\begin{itemize}
\item
  Una pregunta con respuesta.
\item
  ¿Cuál es el efecto de un tratamiento \(T\) sobre una variable \(Y\)?
\item
  Suele ser una cantidad de interés que resume los datos:

  \begin{itemize}
  \item
    Descriptiva: ¿Cuál es la media de \(Y\) en el tratamiento
    formalmente?
  \item
    Causal: ¿Cuál sería la diferencia promedio de \(Y\) si cambiáramos
    el tratamiento por el control? Si afirmamos que \(T\) no tiene
    ningún efecto en \(Y\), ¿qué tanta evidencia tendríamos para hacer
    esta afirmación?
  \item
    La cantidad es el estimando o la hipótesis.
  \end{itemize}
\item
  No todas las preguntas que queremos hacer tienen respuesta.

  \begin{itemize}
  \tightlist
  \item
    Y la gama de indagaciones que podemos hacer es limitada: ¿cuánto
    podemos aprender de una cantidad que resume los datos, como lo es el
    efecto de tratamiento promedio (ATE)?
  \end{itemize}
\end{itemize}
\end{frame}

\begin{frame}{Datos}
\protect\hypertarget{datos}{}
\begin{itemize}
\item
  Recolectar (generar) datos sobre el conjunto de variables (todas:
  \(X\)s, \(T\)s y \(Y\)s)
\item
  Una función del modelo
\item
  Incluye ambos:

  \begin{itemize}
  \item
    Muestreo: ¿cómo llegan las unidades a su muestra?
  \item
    Asignación de tratamiento: ¿qué valores de las variables endógenas
    se expresan?
  \end{itemize}
\end{itemize}
\end{frame}

\begin{frame}{Respuesta (Answer Strategy)}
\protect\hypertarget{respuesta-answer-strategy}{}
\begin{itemize}
\item
  Generar una respuesta dada la recolección de los datos, es decir, una
  estimación de la cantidad de interés (indagación)
\item
  Este es su estimador o prueba:

  \begin{itemize}
  \item
    Diferencia de medias
  \item
    prueba \(t\)
  \item
    Métodos de regresión
  \item
    etc.
  \end{itemize}
\item
  La respuesta es una estimación de la cantidad de interés o valor de
  \(p\) (indagación/estimación /prueba)
\end{itemize}
\end{frame}

\hypertarget{pre-registro-del-plan-de-anuxe1lisis}{%
\section{Pre-registro del plan de
análisis}\label{pre-registro-del-plan-de-anuxe1lisis}}

\begin{frame}{Sesgo en la investigación publicada contra resultados
nulos}
\protect\hypertarget{sesgo-en-la-investigaciuxf3n-publicada-contra-resultados-nulos}{}
\begin{itemize}
\item
  Anticipándose a las dificultades que hay al publicar, o para no
  enfrentarse a ellas, manuscritos con resultados nulos no se envían
  nunca para revisión o se guardan en un ``cajón de archivos'' después
  de varios rechazos.
\item
  Todos enfrentamos incentivos para cambiar nuestras especificaciones,
  medidas o incluso hipótesis para obtener un resultado estadísticamente
  significativo (\(p\)-hacking) para mejorar las posibilidades de
  publicación.
\item
  Incluso cuando no se enfrentan estos incentivos se toman muchas
  decisiones al analizar los datos: manejo de valores perdidos y
  observaciones duplicadas, creación de escalas, etc. Y estas elecciones
  pueden tener consecuencias.
\item
  Resultado general: menor credibilidad en los trabajos de investigación
  individuales y (con razón) menor confianza en si realmente sabemos lo
  que decimos saber.
\end{itemize}
\end{frame}

\begin{frame}{Hacia la revisión del diseño en lugar de los resultados}
\protect\hypertarget{hacia-la-revisiuxf3n-del-diseuxf1o-en-lugar-de-los-resultados}{}
\begin{itemize}
\item
  Una parte de la solución de este problema es centrarse en el diseño,
  más que en los resultados.
\item
  El sesgo contra los resultados nulos se puede combatir revisando el
  diseño antes de conocer los resultados.
\item
  Un buen diseño bien ejecutado producirá una investigación creíble,
  incluso con resultados nulos. Queremos que los resultados nulos sean
  creíbles y procesables.
\item
  La revisión del diseño también es una oportunidad para mejorar la
  investigación antes de que sea implementada.
\end{itemize}
\end{frame}

\begin{frame}{Pre-registro de planes de análisis y diseños de
investigación I}
\protect\hypertarget{pre-registro-de-planes-de-anuxe1lisis-y-diseuxf1os-de-investigaciuxf3n-i}{}
\begin{itemize}
\item
  El pre-registro es incluir el diseño de investigación e hipótesis en
  un repositorio de acceso público. EGAP aloja uno que se puede usar de
  forma gratuita (actualmente en \href{https://osf.io}{OSF.io}
  utilizando el formulario de registro EGAP).
\item
  El pre-registro no implica excluir análisis exploratorios posteriores
  que no fueron registrados con anticipación. Simplemente se debe
  distinguir claramente entre ambos.
\end{itemize}
\end{frame}

\begin{frame}{Pre-registro de planes de análisis y diseños de
investigación II}
\protect\hypertarget{pre-registro-de-planes-de-anuxe1lisis-y-diseuxf1os-de-investigaciuxf3n-ii}{}
\begin{itemize}
\item
  Incluso si se va a enviar un artículo con resultados en lugar de un
  diseño para una revista académica o se está interesado principalmente
  en el informe final con hallazgos para una audiencia en el campo de
  las políticas públicas, existen ventajas importantes para ustedes y
  para otros investigadores del pre-registro de su investigación.

  \begin{itemize}
  \item
    Por ejemplo, se puede obtener información sobre otras
    investigaciones, completadas y en curso; otros pueden aprender sobre
    su trabajo. Podemos aprender de estudios que produjeron resultados
    nulos.
  \item
    Les obliga a plantear sus hipótesis y plan de análisis antes de ver
    los resultados, lo que limita el \(p\)-hacking.
  \end{itemize}
\end{itemize}
\end{frame}

\hypertarget{resumen}{%
\section{Resumen}\label{resumen}}

\begin{frame}{El proceso de investigación: preguntas, teoría y
credibilidad}
\protect\hypertarget{el-proceso-de-investigaciuxf3n-preguntas-teoruxeda-y-credibilidad}{}
\begin{itemize}
\item
  La investigación parte de nuestros valores y teorías sobre cómo
  funciona el mundo.
\item
  Continúa articulando preguntas que se pueden abordar de forma clara
  mediante la observación (en este curso, utilizando experimentación
  aleatoria).
\item
  Las buenas preguntas tienen respuestas consecuentes: cambian las
  explicaciones científicas y/o cambian las decisiones de políticas.
\end{itemize}
\end{frame}

\begin{frame}{El proceso de investigación: preguntas, teoría y
credibilidad}
\protect\hypertarget{el-proceso-de-investigaciuxf3n-preguntas-teoruxeda-y-credibilidad-1}{}
\begin{itemize}
\item
  Los buenos diseños de investigación satisfacen todos los requisitos y
  dan a los lectores razones para creer en los resultados.
\item
  Las listas de verificación (como el formulario de diseño de
  investigación o los formularios de preinscripción) ayudan a evitar
  errores por descuido.
\item
  El pre-registro aumenta aún más la credibilidad y, por lo tanto, las
  probabilidades de que sus resultados tengan un impacto en la ciencia y
  en las políticas.
\end{itemize}
\end{frame}

\hypertarget{referencias}{%
\section{Referencias}\label{referencias}}

\begin{frame}{Referencias}
\protect\hypertarget{referencias-1}{}
\end{frame}

\end{document}
